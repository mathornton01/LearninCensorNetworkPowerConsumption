\documentclass{beamer}
\usepackage{booktabs}
\usepackage{graphicx}
\usepackage{enumerate}
\usepackage{subfigure}
\usepackage{latexsym}
\usepackage{color}
\usepackage{epstopdf}
\usepackage{tikz}
\usepackage{movie15}
\usepackage{amsmath}
\usepackage{import}
%\usepackage[showframe=true]{geometry}
\usepackage{changepage}

\usetikzlibrary{arrows,shapes}


\definecolor{wwu}{rgb}{0,0.11,0.23}
\definecolor{wwumed}{rgb}{0.74,0.76,0.75}
\definecolor{ercisred}{rgb}{0.38,0.12,0.15}
\setbeamercolor{alerted text}{fg=ercisred}
\setbeamercolor{normal text}{fg=wwu}

\usepackage{colortbl}
\usepackage{dcolumn}
\newcolumntype{d}[1]{D{.}{.}{#1}}
\newcolumntype{.}{D{.}{.}{-1}}

\mode<presentation>
{
  \usetheme{Warsaw}
  \usecolortheme{whale}
  \setbeamercovered{transparent}
}


\usepackage[english]{babel}
\usepackage{inputenc}
\usepackage{beamerfontthemeprofessionalfonts}

% ----- macros for result figures ------
\newcommand{\posbar}[2]{\tikz[x=.2cm]{%
	\draw [green,fill] (0,0) rectangle (-#2,.25); 
	\draw (22,.8) node [below left,overlay] {\parbox{3.5cm}{\hphantom{A}\raggedleft\small #1}};
}}

\newcommand{\negbar}[2]{\tikz[x=.2cm]{%
	\draw [red!60,fill] (0,0) rectangle (-#2,.25); 
	\draw (-22,.8) node [below right,overlay] {\parbox{3.5cm}{\hphantom{A}\raggedright\small #1}};
}}

\newcommand{\myscales}{%
\tikz[x=.2cm,>=stealth]{\draw [|->] (0,0)--(-5,0)--(-17,0) node [below left] {\small \%-pts.};
	\draw (-5,.1)--++(0,-.2) node [below] {\scriptsize $-5$};
	\draw (-10,.1)--++(0,-.2) node [below] {\scriptsize $-10$};
	\draw (-15,.1)--++(0,-.2) node [below] {\scriptsize $-15$};
}&
\tikz[x=.2cm,>=stealth]{\draw [|->] (0,0)--(17,0) node [below right] {\small \%-pts.};
	\draw (5,.1)--++(0,-.2) node [below] {\scriptsize $5$};
	\draw (10,.1)--++(0,-.2) node [below] {\scriptsize $10$};
	\draw (15,.1)--++(0,-.2) node [below] {\scriptsize $15$};
}}


%NOTE: The proper title of the paper will not fit on the bottom
%of the slide, this is part of the Warsaw theme, and I am unsure
%of how it can directly be changed.
\title{Power Consumption Attacks in Wireless Sensor Networks}
\author[Micah Thornton \& Ryan Sligh \& Robert Santoski]{Micah Thornton \and {Ryan Sligh} \and Bobby Santoski}
\institute{Computer Science \& Engineering, Southern Methodist University, USA, \texttt{mathornton@smu.edu} \\ \texttt{rsligh@smu.edu} \\ \texttt{rsantoski@smu.edu}}

\date[] 
{CSE 4344: Networks and Distributed Systems\\
Dallas, Texas\\
April 26, 2014}


%begin custom commands for full-size pics
\newenvironment{changemargin}[2]{%
\begin{list}{}{%
\setlength{\topsep}{0pt}%
\setlength{\leftmargin}{#1}%
\setlength{\rightmargin}{#2}%
\setlength{\listparindent}{\parindent}%
\setlength{\itemindent}{\parindent}%
\setlength{\parsep}{\parskip}%
}%
\item[]}{\end{list}}


% Make one image take up the entire slide content area in beamer,.:
% centered/centred full-screen image, with title:
% This uses the whole screen except for the 1cm border around it
% all. 128x96mm
\newcommand{\titledFrameImage}[2]{
\begin{frame}{#1}
\begin{changemargin}{-1cm}{-1cm}
\begin{center}
\includegraphics[width=108mm,height=.88\textheight,keepaspectratio]{#2}
\end{center}
\end{changemargin}
\end{frame}
}

\newcommand{\titledFrameImageURL}[3]{
\begin{frame}{#1}
\begin{changemargin}{-1cm}{-1cm}
\begin{center}
\includegraphics[width=108mm,height=.82\textheight,keepaspectratio]{#2}
\newline
{\tiny Source: \url{#3}}
\end{center}
\end{changemargin}
\end{frame}
}

\newcommand{\titledFrameImageSource}[3]{
\begin{frame}{#1}
\begin{changemargin}{-1cm}{-1cm}
\begin{center}
\includegraphics[width=108mm,height=.82\textheight,keepaspectratio]{#2}
\newline
{\tiny Source: #3}
\end{center}
\end{changemargin}
\end{frame}
}

\newcommand{\titledFrameDoubleImage}[3]{
\begin{frame}{#1}
\begin{changemargin}{-1cm}{-1cm}
\begin{center}
\includegraphics[width=108mm,height=.5\textwidth,keepaspectratio]{#2}
\includegraphics[width=108mm,height=.5\textwidth,keepaspectratio]{#3}
\end{center}
\end{changemargin}
\end{frame}
}


% Make one image take up the entire slide content area in beamer.:
% centered/centred full-screen image, no title:
% This uses the whole screen except for the 1cm border around it
% all. 128x96mm
\newcommand{\plainFrameImage}[1]{
\begin{frame}[plain]
\begin{changemargin}{-1cm}{-1cm}
\begin{center}
\includegraphics[width=108mm,height=76mm,keepaspectratio]{#1}
\end{center}
\end{changemargin}
\end{frame}
}

% Make one image take up the entire slide area, including borders, in beamer.:
% centered/centred full-screen image, no title:
% This uses the entire whole screen
\newcommand{\maxFrameImage}[1]{
\begin{frame}[plain]
\begin{changemargin}{-1cm}{-1cm}
\begin{center}
\includegraphics[width=\paperwidth,height=\paperheight,keepaspectratio]
{#1}
\end{center}
\end{changemargin}
\end{frame}
}
%end custom commands for full-size pics


% create a plus and minus bullet point for advantages/disadvantages
\newcommand{\positem}[1]{
\item[\color{green}{\bf +}] #1}

\newcommand{\negitem}[1]{
\item[\color{red}{\bf -}] #1}
\newcommand{\xitem}[1]{
\item[\color{red}{\bf X}] #1}


\AtBeginSection[]
{
  \begin{frame}<beamer>
    \frametitle{Outline}
    \tableofcontents[currentsection]
  \end{frame}
}

\begin{document}

\begin{frame}
  \titlepage 
\end{frame}

 \begin{frame}{Outline of today's talk}
    \tableofcontents
  \end{frame}

\section{Introduction}
\subsection{Topics}

\begin{frame}{Brief Intro to Wireless Sensor Networks(WSNs)}
\begin{itemize}
	\item A \textbf{wireless sensor network(WSN)} is a network of \textbf{Sensor Nodes}
	\item \textbf{Sensor Nodes} send and receive wide varieties of data.
	\item \textbf{Sensor Nodes} are developed in bulk for mass deployment 
	\item \textbf{Sensor Nodes} operate in one of two states: 
	\begin{itemize}
		\item \textbf{Sleep Mode} - less power draw, can't receive and transmit 
		\item \textbf{Active Mode} - more power draw, can receive and transmit
	\end{itemize}
	\item \textbf{WSNs} can be applied to many problems
\end{itemize}
\end{frame}

\titledFrameImage{WSN examples (1)}{Figures/Portada_490px.jpg}
\titledFrameImage{WSN examples (2)}{Figures/forest_fire_detection.png}
\titledFrameImage{WSN examples (3)}{Figures/security_sensor.png}

\begin{frame}{Attacks on WSN power supplies}
\begin{itemize}
	\item Bulk production has robbed WSNs of more robust \textbf{battery lives}
	\item The nature of WSNs makes them easy targets for \textbf{Power Consumption Attacks}
	\item A \textbf{Power Consumption Attack} exploits the small battery life of Sensor Nodes by draining the battery
	\item This attack can have devastating effects on the WSN
	\item \textbf{Power Consumption Attacks} are performed in multiple ways
\end{itemize}	
\end{frame}

\titledFrameImage{Power Consumption attack models (1)}{Figures/AModel1.png}
\titledFrameImage{Power Consumption attack models (2)}{Figures/AModel2.png}
\titledFrameImage{Power Consumption attack models (3)}{Figures/AModel3.png}

\subsection{Motivation}

\begin{frame}{Problem}
\begin{itemize}

	\item \textbf{How do we defend against a wide range of Power Consumption Attacks?}
	
\end{itemize}
\end{frame}

\section{Methodology}

\subsection{Battery Behavior}

\begin{frame}{Battery Tests}
\begin{itemize}
	\item The logical conclusion to mitigate risks of \textbf{Power Consumption Attacks} is to use more powerful \textbf{batteries}
	\item Another simulation we ran tested various types of batteries
	\item The batteries tested were: 
	\begin{itemize}
	  \item Lead-Acid Batteries
	  \item Alkaline Long-Life Batteries
	  \item Carbon-Zinc Batteries
	  \item NiMH Batteries
	  \item NiCad Batteries
	  \item Lithium Ion Batteries
	\end{itemize}
	\item With weights varying from \textbf{0.1 mg} to \textbf{1 mg}
	\item And Packet sizes varying from \textbf{2 bits} to \textbf{1 kb}
\end{itemize}
\end{frame}

\subsection{Attack Simulations}

\begin{frame}{Attack Simulation}
	\begin{itemize}
	\item The \textbf{standard power consumption attack} seen in model 1 and the \textbf{routing power consumption attack} seen in model 3 were simulated in an environment that allowed user defined: 
	\begin{itemize}
		\item Packet Size (bits)
		\item Initial Node Energy (joules)
		\item Power To Transmit Messages (Watts)
		\item Power To Receive Messages (Watts)
		\item speed of Transmission radios (bps)
	\end{itemize}
	\item Each of these were variate for 55,000 simulations.
	
	\end{itemize}
\end{frame}



\section{Results and Analysis}

	\titledFrameImage{Lithium Ion Results}{Figures/LiIonBATpkt.pdf}
	\titledFrameImage{Comparing Attacks}{Figures/pktszndtmtocmp1.pdf}
\subsection{Simulation Results}

\subsection{Mitigation Strategies}

\begin{frame}{Previous Strategies}

\begin{itemize}

	\item Some \textbf{risk mitigation strategies} have already been adopted for use in WSNs:
	\begin{itemize}
		\item \textbf{Predefined Transfer Windows}
		\item \textbf{Node Reception Memory}
		\item \textbf{Jamming Detection Protocols}
		\item \textbf{Low Power Wake-up Radio}
		\item \textbf{Defined Maximum Path Length}
	\end{itemize}
	\item Many strategies are developed with specific attacks in mind
	\item Even our proposed strategies have already been deployed
	
\end{itemize}

\end{frame}

\begin{frame}{Proposed Strategies}

\begin{itemize}

	\item Targeted the root problem of all Power Consumption attacks: \textbf{pre-defined battery life}
	\item Installation of solar panels and other similar power regeneration devices.
	\item Attacks can still be mounted on the network, but would have to fight a endlessly renewing power source 
	\item This addition could be costly, and distributors would need to shrink the size of their network
	\item But it is up to the distributor to examine there expected net benefit
	
\end{itemize}

\end{frame}

\section{Conclusion}

\subsection{Future Work}
\begin{frame}{Future Work}
\begin{itemize}

	\item Model and test additional attack types 
	\item Do a cost benefit analysis of different types of \textbf{batteries} and \textbf{alternative power sources}
	\item compare cost benefits of other mitigation strategies 

\end{itemize}
\end{frame}	
\end{document}