\section{Conclusion}
\subsection{Future Work}
After running these simulations there are a number of different ways 
we can go with our future work.  One is to add more complexity to 
the simulation in order to gain deeper insight into how a wireless 
network of sensor nodes reacts to a power consumption attack.  We 
could model the entire network of sensor nodes that the target node 
is connected to so that we can see how the whole network reacts to 
one of the nodes losing power. We can also implement and simulate 
different security protocols to see how they stand up to the attack.  
Since we already have the framework for this type of attack we can 
also research, model, and simulate additional attack methods.

We will also be doing a cost benefit analysis of the batteries for individual sensor 
nodes, that shows how likely a distributor is to use a larger battery with the hope of
mitigating power consumption attacks on sensor nodes. We will also run 
similar simulations with different parameters varied, such as distance to transmit, and 
receive and more. 
 
\subsection{Remarks}
In the current phase of our research, we have investigated the effects of variable battery 
types and weights on the time to consume all of a nodes energy. This is important for our future research, 
because it helps provide us with some introductory points. Meaning, we will be able to adapt what we 
have learned up to this point to continue our investigation, and run similar simulations under 
different assumptions. It also helped give us a starting point for the next consumption attack 
we will be modeling, as discussed in the methodology section.  

\subsection{Limitations}
The most substantial limitations to this project were the amount of time we had to work, and the fact that 
there is a steep learning curve with NS3 simulator. Some future limitations we will most certainly be facing 
are: NS3 doesn't support the WLAN protocols we need to use, so we will have to approximate them in NS3 or use a 
different simulator. And again the time frame on this project is very limiting, it is my hope (at least), that 
this project can be carried to a certain point within the semester, and then to fruition in form of a publication 
beyond the end of the semester.