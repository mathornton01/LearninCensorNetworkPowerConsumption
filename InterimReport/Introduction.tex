(Describe why WSN’s are important here)
Power is the most important and most limited resource for wireless sensor networks. In[],Manju identifies the activities which waste battery power the most in a WSN node.  The first is when collision corrupts as transmission, and the retransmissions are required since error correcting codes, which use additional power, are not often used [Raymond]. Also, overheating is when a node is needlessly awakened from sleep mode when it intercepts a transmission that was meant for a different node.  Misuse of control packet [Manju] can also drain power by keeping a node listening until a timeout occurs for packets that will never arrive or causing the a node to needlessly transmit responses to the control packet depending on the protocol. 
	An attacker could take advantage of these power wasting scenarios and force them to happen more often than they normal would.  Any attack that causes these effects for long periods of time are considered Denial-of-Sleep [should probably cite this somewhere] attack because a node is being force to use up power to performing meaningless task instead of switching to a powering conserving sleep mode. Eugene Vasserman also brings up that attackers could 

Paper Content
In this paper, we will be simulating several simple wireless sensor node attacks in order to figure out methods of protecting against such attacks. First we will establish what resources and what knowledge of the protocol the attacker will have for each simulation. Then we will explain the details of the scenario we are simulating, including the node information. We will also describe how the simulation is set up, as well as the assumptions we took while designing them. With the framework of the simulation establish, we will analyze the patterns and implication of the collected data. Then we will use the data analysis to support our proposal for countermeasures against power consumption attacks on wireless sensor networks.

