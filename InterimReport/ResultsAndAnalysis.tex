\section{Results and Analysis}

\subsection{Simulation Details and Result gathering}
It is worth reiterating over and over again that the key premise investigated in this project up to this point is the simply energy consumption attack on a specific node that uses the wifi radio protocols. As of yet we have not adapted our model to make use of the existing WSN protocols
Such as LEACH, ZIGBEE, or other WPAN protocols {Citation necessary}. Because of the limited nature of the NS3 simulation tool that has been used to simulate the power consumption attack, we are looking at problems with adapting any type of IEEE WPAN protocol, and are considering the use of a different simulation tool in our future efforts. 
t

Several factors were investigated by the initial simulation of a Power Consumption attack on WiFi radio (WiFi radios are used in the simulation in order to estimate Tx Current and Rx Current and because they are similar to WSN communications). As mentioned in the set up section of our methodology, we used an attack that caused the target node to receive a lot of communication in a short amount of time (1 packet/ 10 ms). The primary advantage of our simulation is that we can manipulate the initial energy supply of the target node. We simulate the initial energy supply based on a few factors, first: what is the internal acid of the battery, second: what is the weight of the battery, third: the potential difference of the battery is assumed to be a constant 3.6 Volts. the information used in these calculations was taken from {citation necessary}.

We ran the simulation multiple times varying the size of the packet transmitted, and the 
