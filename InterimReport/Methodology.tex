2. Methodology
a. Adversarial Assumptions
	There are a few assumptions that are made about the supposed attacker in the simulation scenario.  The first assumption is that the attacker is connected wirelessly to the target node. Another assumption is that the attacker has knowledge about the node they are attacking. The attackers goal in this scenario is to consume enough of the power from the target node so that the node is no longer able to transmit packages.  Another assumption is that the attacker has the capabilities to send a high volume of packets very quickly to the target node.

b. Target Assumptions
	The target node similarly has assumptions associated with it in the simulation.  We assume that the node is a wireless sensor node connected to other wireless sensor nodes.  We also assume that the most power in the node is allocated to transmission, with reception being second.  We assume that the target node has limited power to allocate to each of its functions.  
c. Simulation Setup
	We are using NS3 to simulate two wireless nodes connected to each other. For the purpose of simplicity and to limit variables we are modeling two nodes, an attacker an a target, as opposed to an entire network of nodes. When setting up the simulation we have multiple variables that we concern ourselves with that can affect the simulation.  The most important variables are the battery size of the target node, the 