\section{Conclusion}

\subsection{Proposed Solutions}
From our results we can draw two definite conclusions. As we have shown earlier there are numerous security protocols for wireless sensor nodes that aim to protect against specific types of power consumption attacks. These are effective for the attacks they are designed to defend against but provide little to no protection against other attacks. If one were to implement security protocols to defend against any attack that is known, the cost of production for any one node quickly rises due to adding additional battery allocation and processing power for each defense. Furthermore this approach forces manufacturers to pick and choose which attacks they will be focusing on defending against, which can leave gaping security holes.  Our first proposed strategy for mitigating power consumption attacks attempts to solve the problem at its most basic level: the power source. The idea of adding more power to sensor nodes is a simple yet effective mitigation strategy as it defends against all types of power consumption attacks.  As shown in our tests, batteries such as the lithium ion can have dramatic effects in increasing the life span of a node when it is the target of a power consumption attack. Additionally, since adding more powerful batteries to every wireless sensor node is not always feasible, the idea of equipping a sensor node with a power regeneration device such as a solar panel or hydroelectric generator is an effective alternative. This has the same effect of adding additional battery with the benefit of having to be replaced much less. Depending on the circumstances of the implementation of any given wireless sensor network either of these methods of increasing battery life could benefit the network's security greatly.

Another conclusion that we can take away from our tests is that the routing power draw attack tested is much more potent than the standard denial of sleep attack tested. Therefore it is important to carefully consider routing procedures in a wireless sensor network. Routing in a wireless sensor network should not be a standard for all sensor networks as the inclusion of it forces a network to either take steps to implement additional security against the attack or open themselves up to devastating attack possibilities. Instead wireless sensor network deployers should examine whether or not routing is neccesary in their network at all. This especially applies to smaller networks that can get by without having to route packets.  Though larger security-conscious wireless sensor networks whose deployers are not able to implement defenses against a routing power draw attack would be better off reconfiguring their network to not have routing nodes.  

\subsection{Future Work}
There are a number of different paths our research can take in the future.  We can model and test additional attack types based on the simulations we have already created. These simulations have shown to give reliable results so it would be fairly easy to add new attack types. We can also do a cost benefit analysis of different types of batteries and alternative power sources for sensor nodes. For example we could compare the relative security benefit of a 1mg lithium ion battery with the security benefit of a small solar panel.  From there we could find out which what is solution is most effective and the circumstances that can either add or detract from its effectiveness. Additionally we can compare the data we gather from the initial cost benefit anaylsis of the solutions to the cost benefit of previously introduced mitigation strategies.  For instance we could analyze the security benefit of a low power wake up radio, such as the one proposed in the paper ``Fighting Insomnia: a Secure Wake-up Scheme for Wireless Sensor Networks" by by R. Falk et al.\cite{5211020} Lastly we can pursue additional research related to the neccesity of routing in wireless sensor networks and its alternatives.
